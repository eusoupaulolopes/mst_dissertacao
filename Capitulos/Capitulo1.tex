% Introdução - vFinal 1.0 - 04-07-2018
\chapter{Introdução}
\label{cap:cap1}

\textbf{Neste capítulo serão colocados textos de exemplo ou indicações para a \textit{contrução}} de uma Dissertação de mestrado em LateX. Uma parte será voltada à estrutura do documento e questões específicas relacionadas à ciência, e outra será dedicada a comandos simples e ``tricks" usados na construção do meu documento original.

Todo este template é apenas uma modularização e tentativa de simplificação do modelo disponível em \url{https://github.com/abntex/abntex2/wiki/Download}. Caso eu esqueça ou algum detalhe passe em branco, a dissertação inteira está disponível em \url{https://v1.overleaf.com/read/gpkgdnttndgf}.

%Acredito também que este modelo sirva para outros programas, mas seu direcionamento %principal, como já citado, é para o PPgSW - IMD - UFRN \cite{xiao_study_2011}. %citação incluída para nao deixar as referencias em branco

\section{Aqui vai uma seção da Introdução}

\section{Sobre o \LaTeX}


Como os dispositivos IoT energy driven podem ser categorizados em uma taxonomia eficaz com base em suas características de consumo de energia?

Quais são as estratégias de throttling mais de acordo com o papel realizado pelos dispositivos IoT energy driven e como elas podem ser aplicadas de forma eficaz?

Quais são os principais desafios e oportunidades associados à implementação da taxonomia proposta em função da aplicação de throttling em IoT energy driven?