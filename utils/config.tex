% \documentclass[
% 	% -- opções da classe memoir --
% 	12pt,				% tamanho da fonte
% 	openright,			% capítulos começam em pág ímpar (insere página vazia caso preciso)
% 	oneside,			% para impressão em verso e anverso. Oposto a oneside
% 	a4paper,			% tamanho do papel. 
% 	% -- opções da classe abntex2 --
% 	%chapter=TITLE,		% títulos de capítulos convertidos em letras maiúsculas
% 	%section=TITLE,		% títulos de seções convertidos em letras maiúsculas
% 	%subsection=TITLE,	% títulos de subseções convertidos em letras maiúsculas
% 	%subsubsection=TITLE,% títulos de subsubseções convertidos em letras maiúsculas
% 	% -- opções do pacote babel --
% 	english,			% idioma adicional para hifenização
% 	french,				% idioma adicional para hifenização
% 	spanish,			% idioma adicional para hifenização
% 	brazil				% o último idioma é o principal do documento
% 	]{abntex2}



% ---
% PACOTES BASICOS
% ---
\usepackage{lmodern}			% Usa a fonte Latin Modern			
\usepackage[T1]{fontenc}		% Selecao de codigos de fonte.
\usepackage[utf8]{inputenc}		% Codificacao do documento (conversão automática dos acentos)
\usepackage{lastpage}			% Usado pela Ficha catalográfica
\usepackage{indentfirst}		% Indenta o primeiro parágrafo de cada seção.
\usepackage{color}				% Controle das cores
\usepackage{graphicx}			% Inclusão de gráficos
\usepackage{microtype} 			% para melhorias de justificação
\usepackage{setspace}           % Uso de espaçamento simples
\usepackage{blindtext}
% ---
% PACOTES ADICIONAIS, USADOS APENAS NO ÂMBITO DO MODELO CANÔNICO DO abnteX2
% ---
\usepackage{lipsum}				% para geração de dummy text

% ---
% PACOTES DE CITAÇÕES
% ---
\usepackage[brazilian,hyperpageref]{backref}	 % Paginas com as citações na bibl
\usepackage[alf]{abntex2cite}	% Citações padrão ABNT

% ---
% PACOTES ADICIONADOS POR CEPHAS
% ---
\usepackage{float}
\usepackage{amssymb,amsmath}
\usepackage{pdfpages}
\usepackage{acronym}

% ---
% PACOTES ADICIONADOS POR PAULO
% ---
\usepackage{subcaption}
\usepackage{threeparttable}
\usepackage{multirow}
\usepackage{arydshln}
\usepackage{rotating}
\usepackage[table]{xcolor}
% --- 
% CONFIGURAÇÕES DE PACOTES
% --- 

% Configurações do pacote backref
% Usado sem a opção hyperpageref de backref
\renewcommand{\backrefpagesname}{Citado na(s) página(s):~}
% Texto padrão antes do número das páginas
\renewcommand{\backref}{}
% Define os textos da citação
\renewcommand*{\backrefalt}[4]{
	\ifcase #1 %
		Nenhuma citação no texto.%
	\or
		Citado na página #2.%
	\else
		Citado #1 vezes nas páginas #2.%
	\fi}%
	

\usepackage{enumitem}
\newcommand{\subscript}[2]{$#1 _ #2$}
% ---

\newcommand\muL{\si{\micro\liter}}
\usepackage{pifont}

\newcommand*\rot{\rotatebox{90}}
\newcommand*\OK{\ding{51}}

